\documentclass[11pt,a4paper]{article}

%Paquetes importados
\usepackage[utf8]{inputenc}
\usepackage[spanish]{babel}
\usepackage{graphicx}
\usepackage{amssymb}
\usepackage{amsmath}
\usepackage{float}
\usepackage{listings}
\usepackage[rgb,svgnames,table]{xcolor}
\usepackage{pdfpages}
\usepackage{fancyhdr}
\usepackage{lastpage}
\usepackage{afterpage}

\addto\captionsspanish{
	\renewcommand\tablename{Tabla}
	\renewcommand\listtablename{Lista de tablas}
	\renewcommand\lstlistingname{Código}
	\renewcommand\lstlistlistingname{Lista de código}
}

\lstset{ % Defino el formato de bloques de código fuente
	inputencoding=utf8, % Indico la codificación de los archivos de entrada
	extendedchars=true, % Extiendo los caracteres
	% Escapeo caracteres especiales
	literate={á}{{\'a}}1 {é}{{\'e}}1 {í}{{\'i}}1 {ó}{{\'o}}1 {ú}{{\'u}}1 {ñ}{{\~n}}1,
	showtabs=false, % Indica si se muestran los tabs
	tabsize=2, % Indica la cantidad de espacios que ocupa un tab
	showspaces=false, % Indica si muestra los espacios
	showstringspaces=false, % Indica si muestra los espacios dentro de strings
	numbers=left, % Posición en que se muestran los números de línea
	numberstyle=\tiny\color{gray}, % Estilo de los números de línea
	breaklines=true, % Se parten las líneas que exceden del ancho del documento
	frame=single, % Formato del marco del entorno
	backgroundcolor=\color{gray!5}, % Color de fondo
	basicstyle=\ttfamily\footnotesize, % Estilo de base (familia, tamaño, color)
	keywordstyle=\color{DarkBlue}, % Estilo de las palabras reservadas
	stringstyle=\color{red}, % Estilo de los strings
	commentstyle=\color{DarkGreen}, % Estilo de los comentarios
	language=Octave, % Espeficica el lenguaje a usar
	otherkeywords={std,cout} % Agrego palabras reservadas que no se resaltan
}

\graphicspath{{imagenes/}}

\newcommand\blankpage{%
	\null
	\thispagestyle{empty}%
	\addtocounter{page}{-1}%
	\newpage}


\pagestyle{fancy}

\lhead[]{$2^{do}$ Cuatrimestre 2017\\TP 0}
\chead[]{}
\rhead[]{\includegraphics[scale=0.2]{logo_fiuba2}}

\lfoot[]{}
\cfoot[]{}
\rfoot[]{\thepage / \pageref{LastPage}}

\begin{document}
\title{TP0 Organización de Computadoras}
\author{Federico Rodriguez Longhi}		
\date{\today}

\begin{titlepage}
	
	\begin{figure}[H]
		\raggedright
		\includegraphics[scale=0.25]{logo_fiuba2}
		\hfill
		\raggedleft
		\includegraphics[scale=0.2]{logo_uba}
	\end{figure}
	\rule{\textwidth}{1pt}\par % Thick horizontal line
	\vspace{2pt}\vspace{-\baselineskip} % Whitespace between lines
	\rule{\textwidth}{0.4pt}\par % Thin horizontal line
	
	\vspace{0.05\textheight} % Whitespace between the top lines and title
	\centering % Center all text
	{\Huge UBA FACULTAD DE INGENIERÍA}\\[0.5\baselineskip]
	{\Large 66.20 Organización de Computadoras}\\[0.5\baselineskip]
	{\Huge Trabajo Práctico 0}\\[0.75\baselineskip]
	{\Large 2$^{do}$ Cuatrimestre 2017}\\[0.5\baselineskip]
	\vspace{0.2\textheight}


	\begin{flushleft}
	\begin{table}[H]
		\begin{flushleft}
		\textbf{Integrantes:}\\
		\vspace{0.01\textheight}
		\begin{tabular}{l r}
			Rodriquez Longhi, Federico  & 93336\\
			\hspace{0.05\textheight}federico.rlonghi@gmail.com&\\
			Deciancio, Nicolás Andrés   & 92150\\
			\hspace{0.05\textheight}nicodec\_89@hotmail.com&\\
			Marshall, Juan Patricio & 95471\\
			\hspace{0.05\textheight}juan.pmarshall@gmail.com&\\
		\end{tabular}
		\end{flushleft}
	\end{table}
		

	\end{flushleft}
	\vspace{0.05\textheight}
	\vspace{2pt}
	\vfill
	\rule{\textwidth}{1pt}\par % Thick horizontal line
	\vspace{2pt}\vspace{-\baselineskip} % Whitespace between lines
	\rule{\textwidth}{0.4pt}\par % Thin horizontal line
	
\end{titlepage}

\blankpage

\tableofcontents

\newpage

\section{Introducción}
El trabajo práctico consistió en la elaboración de un programa escrito en lenguaje C, el cual consistía en el procesamiento de texto para determinar palabras palíndromas dentro del mismo.
El código fue ejecutado sobre el sistema operativo linux y netbsd (provisto por el curso). Dentro del ambiente virtual se compilo el código para obtener la salida en código MIPS.

\section{Documentación}
El uso del programa se compone de las siguientes opciones que le son pasadas por parámetro:
\begin{itemize}
	\item \texttt{-h} o \texttt{--help}: muestra la ayuda.
	\item \texttt{-V} o \texttt{--version}: muestra la versión.
	\item \texttt{-i} o \texttt{--input}: recibe como parámetro un archivo de texto como entrada. En caso de que no usar esta opción, se toma como entrada la entrada estándar.
	\item \texttt{-o} o \texttt{--output}: recibe como parámetro un archivo de texto como salida. En caso de que no usar esta opción, se toma como salida la salida estándar.
\end{itemize}

\section{Compilación}
El programa puede ser compilado ubicándose en la carpeta que contiene el código fuente tp0.c y correr el siguiente comando:

\texttt{gcc -Wall -o "tp0" "tp0\_2.c"}\\

También se provee de un script \texttt{compilar} el cual nos compilará el código automáticamente:

\texttt{./compilar}


\section{Pruebas}
Para las pruebas se proveen de dos scripts que las ejecutan.
El primer script \texttt{test.sh} ejecuta los ejemplos del enunciado.

El segundo script \texttt{test\_p.sh} ejecuta las pruebas propias.
Este archivo esta diseñado para poder agregar pruebas de forma sencilla, simplemente se debe agregar una linea en el sector de pruebas de la siguiente manera:\\

\texttt{make\_test <nombre> <entrada de texto> <salida esperada>}

Este script crea los archivos correspondientes en la carpeta tests (dentro del directorio sobre el cual se ejecuta).
Los archivos creados son de la forma:

\begin{itemize}
	\item \texttt{test-<nombre del test>\_in}: archivo de entrada
	\item \texttt{test-<nombre del test>\_out}: archivo de salida generado por el programa
	\item \texttt{test-<nombre del test>\_expected}: archivo de salida esperado
\end{itemize}

\subsection{Corridas de Prueba}
A continuación se muestran las corridas de prueba generadas por el script:

\section{Conclusión}
Al realizar este trabajo, comprendimos todo lo que implica simular un sistema operativo y lograr un entorno de desarrollo estable para trabajar a nivel MIPS. La apertura de un túnel reverso parecía que iba a ser una dificultad, pero con la documentación dada en la práctica no lo fue. Ayudo que todos los integrantes del grupo teníamos alguna experiencia en el uso de sistemas operativos UNIX. 
Gracias a  los flags de compilación vistos, pudimos obtener el código assembly de nuestro programa escrito en C. Nos resulto muy interesante que un programa no muy complejo y de pocas lineas, convertido a assembly ocupara tantas lineas. Es lógico, viendo que C es un lenguaje de un nivel mucho mas alto que assembly, y se abstrae de muchos elementos de la arquitectura de la computadora, que en assembly son más relevantes.
Como conclusión grupal, llegamos a que este práctico nos sirvió como iniciación a todo un mundo de desarrollo a bajo nivel, usando codigo MIPS, que previamente pasabamos bastante por alto.

\begin{lstlisting}

Compiling Source
Compilation Success
Starting Tests

Test: one_letter_a
Test passed

Test: empty_file
Test passed

Test: no_palindroms
Test passed

Test: todos_palindromos
Test passed

Test: varias_lineas
Test passed

Test: all_letters
Test passed

Test: case_sensitive
Test passed

Test: numbers_and_letters
Test passed

Test: text_with_dash
Test passed

--------------------
All 9 tests passed!!!
--------------------
\end{lstlisting}



\section{Código en C}

\lstinputlisting[language=C]{tp0.c}

\section{Código en MIPS}
\lstinputlisting[language=C]{tp0.s}

\section{Enunciado}
El enunciado se encuentra anexado al final de este documento.



\end{document}
