\documentclass[11pt,a4paper]{article}

%Paquetes importados
\usepackage[utf8]{inputenc}
\usepackage[spanish]{babel}
\usepackage{graphicx}
\usepackage{amssymb}
\usepackage{amsmath}
\usepackage{float}
\usepackage{listings}
\usepackage[rgb,svgnames,table]{xcolor}
\usepackage{pdfpages}
\usepackage{fancyhdr}
\usepackage{lastpage}
\usepackage{afterpage}

\addto\captionsspanish{
	\renewcommand\tablename{Tabla}
	\renewcommand\listtablename{Lista de tablas}
	\renewcommand\lstlistingname{Código}
	\renewcommand\lstlistlistingname{Lista de código}
}

\lstset{ % Defino el formato de bloques de código fuente
	inputencoding=utf8, % Indico la codificación de los archivos de entrada
	extendedchars=true, % Extiendo los caracteres
	% Escapeo caracteres especiales
	literate={á}{{\'a}}1 {é}{{\'e}}1 {í}{{\'i}}1 {ó}{{\'o}}1 {ú}{{\'u}}1 {ñ}{{\~n}}1,
	showtabs=false, % Indica si se muestran los tabs
	tabsize=2, % Indica la cantidad de espacios que ocupa un tab
	showspaces=false, % Indica si muestra los espacios
	showstringspaces=false, % Indica si muestra los espacios dentro de strings
	numbers=left, % Posición en que se muestran los números de línea
	numberstyle=\tiny\color{gray}, % Estilo de los números de línea
	breaklines=true, % Se parten las líneas que exceden del ancho del documento
	frame=single, % Formato del marco del entorno
	backgroundcolor=\color{gray!5}, % Color de fondo
	basicstyle=\ttfamily\footnotesize, % Estilo de base (familia, tamaño, color)
	keywordstyle=\color{DarkBlue}, % Estilo de las palabras reservadas
	stringstyle=\color{red}, % Estilo de los strings
	commentstyle=\color{DarkGreen}, % Estilo de los comentarios
	language=Octave, % Espeficica el lenguaje a usar
	otherkeywords={std,cout} % Agrego palabras reservadas que no se resaltan
}

\graphicspath{{imagenes/}}

\newcommand\blankpage{%
	\null
	\thispagestyle{empty}%
	\addtocounter{page}{-1}%
	\newpage}


\pagestyle{fancy}

\lhead[]{$2^{do}$ Cuatrimestre 2017\\TP 1}
\chead[]{}
\rhead[]{\includegraphics[scale=0.2]{logo_fiuba2}}

\lfoot[]{}
\cfoot[]{}
\rfoot[]{\thepage / \pageref{LastPage}}

\begin{document}
\title{TP1 Organización de Computadoras}
\author{Federico Rodriguez Longhi}		
\date{\today}

\begin{titlepage}
	
	\begin{figure}[H]
		\raggedright
		\includegraphics[scale=0.25]{logo_fiuba2}
		\hfill
		\raggedleft
		\includegraphics[scale=0.2]{logo_uba}
	\end{figure}
	\rule{\textwidth}{1pt}\par % Thick horizontal line
	\vspace{2pt}\vspace{-\baselineskip} % Whitespace between lines
	\rule{\textwidth}{0.4pt}\par % Thin horizontal line
	
	\vspace{0.05\textheight} % Whitespace between the top lines and title
	\centering % Center all text
	{\Huge UBA FACULTAD DE INGENIERÍA}\\[0.5\baselineskip]
	{\Large 66.20 Organización de Computadoras}\\[0.5\baselineskip]
	{\Huge Trabajo Práctico 1}\\[0.75\baselineskip]
	{\Large 2$^{do}$ Cuatrimestre 2017}\\[0.5\baselineskip]
	\vspace{0.2\textheight}


	\begin{flushleft}
	\begin{table}[H]
		\begin{flushleft}
		\textbf{Integrantes:}\\
		\vspace{0.01\textheight}
		\begin{tabular}{l r}
			Rodriquez Longhi, Federico  & 93336\\
			\hspace{0.05\textheight}federico.rlonghi@gmail.com&\\
			Deciancio, Nicolás Andrés   & 92150\\
			\hspace{0.05\textheight}nicodec\_89@hotmail.com&\\
		\end{tabular}
		\end{flushleft}
	\end{table}
		

	\end{flushleft}
	\vspace{0.05\textheight}
	\vspace{2pt}
	\vfill
	\rule{\textwidth}{1pt}\par % Thick horizontal line
	\vspace{2pt}\vspace{-\baselineskip} % Whitespace between lines
	\rule{\textwidth}{0.4pt}\par % Thin horizontal line
	
\end{titlepage}

\blankpage

\tableofcontents

\newpage

\section{Introducción}

Este trabajo práctico, consistió principalmente en la codificación de la función palíndromo en código mips, implementada previamente en el tp0 en lenguaje C. Además, se reestructuro el método de input/output del programa para poder limitar la cantidad de acceso a escritura y lectura. Esto se logró con la utilización de buffers, tanto de escritura como de lectura. Los cuales se leen/escriben bien cuando se llenan o cuando se llego al final de la entrada y no hay que analizar más palabras.

Con esto tomamos como objetivo también poder hacer un análisis del costo de las llamadas a sistema. La idea es ver cuanto pierde un programa en performance al realizar estas llamadas.

Todo esto, nos obligó a interiorizarnos mucho más con el conjunto de instrucciones assembly para mips 32, y el uso de syscalls. Además, vimos de forma práctica, como llamar a funciones de archivos .s desde codigo C. El uso de un debugger fue de gran importancia para todo el desarrollo del prorgama en mips.
A continuación, se ve la documentación del programa, forma de compilarlo, casos de prueba armados, análisis del problema y su solución, y el código fuente del programa.

\section{Documentación}
El uso del programa se compone de las siguientes opciones que le son pasadas por parámetro:
\begin{itemize}
	\item \texttt{-h} o \texttt{--help}: muestra la ayuda.
	\item \texttt{-V} o \texttt{--version}: muestra la versión.
	\item \texttt{-i} o \texttt{--input}: recibe como parámetro un archivo de texto como entrada. En caso de que no usar esta opción, se toma como entrada la entrada estándar.
	\item \texttt{-o} o \texttt{--output}: recibe como parámetro un archivo de texto como salida. En caso de que no usar esta opción, se toma como salida la salida estándar.
	\item \texttt{-I} o \texttt{--ibuf-bytes}: recibe como parámetro el tamaño en bytes del buffer de entrada.
	\item \texttt{-O} o \texttt{--iobuf-bytes}: recibe como parámetro el tamaño en bytes del buffer de salida.
	
\end{itemize}

\subsection{Diagrama de Proceso}
A continuación se muestra un diagrama del proceso que recorre el programa, haciendo mención de los buffers que se utilizan y las acciones de comunicación de datos entre ellos.

\begin{figure}[H]
	\centering	
	\includegraphics[width=\textwidth]{diagra_proceso}
	\caption{Diagrama del proceso de comunicación entre buffers}
\end{figure}


\section{Compilación}
Dentro del directorio raiz se encuentra un Makefile. Ejecuntando \texttt{make} se compilara el programa y se generará el ejecutable \texttt{tp1}.

\section{Pruebas}
Para las pruebas se proveen de dos scripts que las ejecutan.
Se utilizaron las mismas pruebas que para el tp0, solo que fueron corridas varias veces variando los parametros -O y -I del programa.

El primer script \texttt{test.sh} ejecuta los ejemplos del enunciado.

El segundo script \texttt{test\_p.sh} ejecuta las pruebas propias.
Este archivo esta diseñado para poder agregar pruebas de forma sencilla, simplemente se debe agregar una linea en el sector de pruebas de la siguiente manera:\\

\texttt{make\_test <nombre> <entrada de texto> <salida esperada>}

Este script crea los archivos correspondientes en la carpeta tests (dentro del directorio sobre el cual se ejecuta).
Los archivos creados son de la forma:

\begin{itemize}
	\item \texttt{test-<nombre del test>\_in}: archivo de entrada
	\item \texttt{test-<nombre del test>\_out}: archivo de salida generado por el programa
	\item \texttt{test-<nombre del test>\_expected}: archivo de salida esperado
\end{itemize}

\subsection{Corridas de Prueba}
A continuación se muestran las corridas de prueba generadas por el script:

\begin{lstlisting}

Compiling Source
Compilation Success
Starting Tests

Test: one_letter_a
Test passed

Test: empty_file
Test passed

Test: no_palindroms
Test passed

Test: todos_palindromos
Test passed

Test: varias_lineas
Test passed

Test: all_letters
Test passed

Test: case_sensitive
Test passed

Test: numbers_and_letters
Test passed

Test: text_with_dash
Test passed

--------------------
All 9 tests passed!!!
--------------------
\end{lstlisting}

\section{Análisis}
Hemos utilizado este programa para analizar los tiempos de las llamadas al kernel. Las syscalls son llamadas al sistema, lo que requiere un cambio de entorno (de modo usuario a modo kernel), y esto consume mayor tiempo que si no se hiciese un cambio de entorno.

Como se provee un buffer de entrada y de salida al modificar el tamaño de estos, estamos disminuyendo o aumentando los syscalls read y write, es decir, cuanto mayor el tamaño del buffer menor cantidad de syscalls y viceversa.

Lo que esperamos obtener como resultado es que al tener un buffer pequeño (por ejemplo de 1 byte) la cantidad de syscalls (la cantidad de syscalls read va a ser la cantidad de letras(bytes) que tenga el archivo) sea mucha, y por lo tanto obtendremos un tiempo de ejecución mayor que el tiempo de ejecucion del mismo programa pero con un buffer mayor (por ejemplo de 1000 bytes, si el archivo es menor o igual al tamaño del buffer haremos un solo read).
El costo de tener un buffer de mayor tamaño es el espacio utilizado por el programa. Esta claro que cuanto mayor sea el buffer mayor sera la memoria requerida por el programa.

Para realizar este análisis utilizamos un archivo de texto como entrada de un tamaño de 317.520 bytes, el cual consiste de muchos palindromos para que el programa utilice la syscall de write. Luego medimos los tiempos de ejecucion del programa. Para ello se provee con un script llamado \texttt{time\_analisis}.
A continuación se muestra una tabla con los resultados:

\begin{table}[h]
	\centering
	\textbf{Tiempos de ejecucion}\\
	\begin{tabular}{|c|c|c|}
		\hline
		Tamaño Buffer Entrada & Tamaño Buffer Salida & Tiempo Ejecucion \\
		\hline
		1 & 1 & 15.918s\\
		\hline
		10 & 10 & 2.352s\\
		\hline
		100 & 100 & 0.566s\\
		\hline
		1000 & 1000 & 0.402s\\
		\hline
		10 & 1 & 10.348s\\
		\hline
		100 & 1 & 9.531s\\
		\hline
		1000 & 1 & 9.375s\\
		\hline
		1 & 10 & 7.699s\\
		\hline
		1 & 100 & 6.652s\\
		\hline
		1 & 1000 & 6.531s\\
		\hline
	\end{tabular}
\end{table}

Se ve claramente como las ejecuciones con buffers mas grandes tardan un tiempo considerablemente menor al que los que tienen buffers de menor tamaño. Siendo el más lento el de buffers de tamaño 1 y 1 y el más rápido el de tamaño 1000 y 1000.

Por lo tanto el analisis cumple con lo que predijimos anteriormente.

\section{Conclusión}
De este trabajo práctico nos llevamos principalmente que es muy importante la experiencia y conocimiento de las instrucciones de Assembly para mips 32 y como usarlas eficientemente. También, pudimos ver por nuestros propios ojos como puede llegar a afectar al tiempo de ejecución de un programa la cantidad de syscalls que se realizan. Acotando las syscalls de read y de write con la utilización de un buffer de escritura y otro de lectura, vimos un speed up global del programa.
Como dijimos previamente, cada syscall implica un cambio de entorno de modo usuario a modo kernel. Y si se repite consistemente este 'switch', el tiempo de ejecución del programa se ve fuertemente afectado. Por esto, siempre que se pueda acotar la cantidad de llamadas al sistema de un programa, se debería tratar de optimizarlas. En este caso particular, se acotó esa cantidad usando los buffers. 
Fue muy importante también el uso de un debugger para poder arreglar problemas durante la codificación.

\appendix

\section*{Apéndice}

\section{Stacks de Funciones}
A continuación se muestran como fueron conformados los stacks de las funciones programadas en MIPS segun lo dispuesto por la ABI.

\subsection{Stack palindromo}
\begin{figure}[H]
	\centering	
	\includegraphics[width=0.6\textwidth]{stack_palindromo}
	\caption{Stack de la función \texttt{int palindromo(int, size\_t, int, size\_t)}}
\end{figure}

\subsection{Stack testpal}
\begin{figure}[H]
	\centering	
	\includegraphics[width=0.6\textwidth]{stack_testpal}
	\caption{Stack de la función \texttt{int testpal(char *, size\_t)}}
\end{figure}




\section{Código en C}
El siguiente código corresponde al cuerpo principal del programa. Se encarga de procesar los parámetros, abrir los archivos, llamar a la función palíndromo (escrita en MIPS) y luego cerrar los archivos correspondientes.

\lstinputlisting[language=C]{code/tp1.c}

\section{Función Palíndromo en MIPS}
Esta función escrita en MIPS es la que corresponde a la pedida en el enunciado, en ella se llama a mymalloc y a testpal (descripta en la próxima sección).

\lstinputlisting[language=C]{code/palindromo.S}

\section{Función testpal}
Hemos desarrollado una función testpal.S la cual recibe en a0 la dirección de la palabra a evaluar, en a1 la longitud de la palabra y devuelve en v0: 0 si la palabra es palíndromo o 1 si no lo es.

\lstinputlisting[language=C]{code/testpal.S}

\section{Función testpal en C}
La siguiente función es el equivalente en C de testpal.S

\lstinputlisting[language=C]{code/testpal.c}

\section{Enunciado}
El enunciado se encuentra anexado al final de este documento.

\end{document}

